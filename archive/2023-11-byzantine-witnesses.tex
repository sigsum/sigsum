\documentclass[a4paper]{article}
\usepackage[utf8]{inputenc}
\usepackage{amsmath}

\newcommand{\floor}[1]{\lfloor #1 \rfloor}
\newcommand{\ceil}[1]{\lceil #1 \rceil}

\author{Niels Möller}
\title{Notes on the implication of dishonest witnesses}

\begin{document}
\maketitle

\section{Introduction}

These notes document the implications of dishonest witnesses. The
objective is to find under which conditions a split-view attack can
succeed, if the log and some of the witnesses are colluding. Or
conversely, assuming a $k$-of-$n$ policy is used, specify how many
honest witnesses are required.

A split-view attack is successful if the log can give two parties two
distinct and inconsistent tree heads, and have both tree heads
accepted.

An honest witness is assumed to sign any tree head the logs sends it,
as long as that is consistent with previously signed tree heads. And
the only property needed for this note is that an honest witness will
never cosign two inconsistent tree heads. We assume that th attacker
chooses which view of the log each honest witness signs.

We assume a compromised or dishonest witness is willing to sign any
tree heads choosen by the attacker, and then cosignatures on different
inconssitent tree heads can be passed to different parties. This setup
is one kind of Byzantine fault tolerance.

\section{Threshold policy}

Assume we have two parties, e.g, a Sigsum monitor and a Sigsum
verifier, using the same list of $n$ witnesses, and where policy
requires at least $k_1$ and $k_2$ valid cosignatures, respectively.

First note that we must require $k_1 + k_2 > n$, otherwise we are
vulnerable to split view attack even if all witnesses are honest.

If we assume at most $f$ of the witnesses are compromized, there can
be up to $n + f$ cosignatures in total on the two views, where the
compromized witnesses sign both views, and the attacker choses which
view is signed by each of the honest witnesses. We then have split
view protection provided that

\begin{equation}
  k_1 + k_2 > n + f. \label{eq:majority-condition}
\end{equation}


If we want to have a simple-to understand symmetric policy where
$k_1 = k_2 = k$, and we only have to assume that $k$ witnesses are
honest, then $f = n - k$, this condition boils down to
\begin{equation}
  k > 2n/3.  %% f = n - k < n - 2n/3 = n/3 < k/2
\end{equation}
Readers familiar with byzantine fault tolerance will recognize this
number. For those of us that are not familiar with that problem, the
condition may be surprisingly strict. E.g, the smallest confiuration
of this type that can tolerate a single compromized witness is 3-of-4,
and the smallest that can tolerate 2 compromized witnesses is 5-of-7.

So with a small number of witnesses, we need rather high availability.

\section{The case for asymmetric policy}

When one of the parties is the verifier, with threshold $k_1$, and the
other is a monitor, with threshold $k_2$, it may be beneficial with
$k_2 > k_1$. The reason is that witnesses can be temporarily
unavailable, so even if a witness is healthy it may not be able to
cosign each and every tree head published by a log. Maybe for some
reason it cosigns only every third tree head. A verifiers only sees
one tree head, and is therefore likely to miss some cosignatures due
to such temporary issues.

A monitor, on the other hand, repeatedly requests the log's tree head,
and checks consistency between sucessive tree heads. It can keep track
of the time\-stamp of the most recent cosignature seen for each witness.

E.g., assume that the monitor has a 12 hour window before alerting on
missing cosignatures. That means that at each point it time, it can
count how many of the witnesses have cosigned a tree head in the last
12 hours. Since it also checks consistency, it can consider the tree
head from 12 hours ago as cosigned by all those witnesses, and if the
number is at least $k_2$, the monitor is happy and doesn't not raise
any alert. In this way, the monitor is shielded from temporary
problems, or witnesses that for some reason cosigns only some of the
tree heads.

We can make use of this monitor advantage to define a \emph{dual
  plocy}. Assume that the verifier uses a $k_1$-of-$n$ policy, and the
intention is to tolerate up to $f$ compromised witnesses (we must have
$f < k_1$). If we apply \eqref{eq:majority-condition} to this case, we
find $k_2 > n + f - k_1$. The dual policy to be used by the monitor
would then be
\begin{equation}
  k_2 = n + f + 1 - k_1
\end{equation}
Unfortunately, this depends on $f$ which must in some way be specified
by the user. It's not so user friendly to have a $f$-of-$k$-of-$n$
policy which would mean that we have $n$ witnesses, require $k$ valid
cosignatures when verifying a sigsum proof, and want to allow for up
to $f$ compromised witnesses.

It's tempting to derive $f$ from $k_1$ by some fix rule. But, e.g., if
we set $f = k_1 - 1$, then that leads to $k_2 = n$, i.e., monitor must
alert if any cosignature is missing over its observation period. Such
a configuration could perhaps be useful in some cases, e.g., if one
can easily revoke or detrust witnesses from verifier policy if they
appear to go offline, but doesn't look very useful for the general
case.

One could maybe also set $f = \ceil{k_1/2} - 1$, motivated by the
byzantine fault tolerance condition when $k \approx 2n/3$. With this
rule, we get
\begin{equation}
  k_2 > n + \ceil{k_1/2} - 1 - k_1 = n - 1 - \floor{k_1/2}
\end{equation}
and the dual policy would be
\begin{equation}
  k_2 = n - \floor{k_1/2}
\end{equation}
This give a family of reasonable pairs of policy and dual policy, with
some examples in Figure~\ref{fig:dual-policies}. Maybe this is good
enough to document and apply automatically when monitoring, to not
require users to specify $f$ explicitly?

\begin{figure}[t]
  \centering
  \begin{tabular}{lll}
    policy & dual & $f$, \# compromized witnesses \\
    $3$-of-n & $(n-1)$-of-$n$ & 1 \\
    $4$-of-n & $(n-2)$-of-$n$ & 1 \\
    $5$-of-n & $(n-2)$-of-$n$ & 2 \\
    $6$-of-n & $(n-3)$-of-$n$ & 2 \\
    $7$-of-n & $(n-3)$-of-$n$ & 3 \\
    \vdots & \vdots & \vdots
  \end{tabular}
  \caption{Dual policies}
  \label{fig:dual-policies}
\end{figure}

\end{document}

%%% Local Variables:
%%% mode: latex
%%% TeX-master: t
%%% End:
